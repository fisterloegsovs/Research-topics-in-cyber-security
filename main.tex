\documentclass{article}
\usepackage{graphicx} % Required for inserting images

\title{Cyber Deception: litterature review}
\author{Anders Friis Persson - s233469\\Malthe Sohansen (s233477) \\ Betim Hudai (s233476) \\  Magnus Frederiksen - s185383}
\date{October 2024}

\begin{document}

\maketitle

\section{Noter}
- Systematic Litterature Review (32 sider)
- Brainstorm
- EMNE: Cyber deception
- Proffesorer vi kan snakke med? (emmanuel, network security forlæsere
- Google event med cyber deception 
- Tror det er nok bare at læse de tekster vi har og så tage underemner hver
- Lav skabelon for reporten

Tidsintervaler magnus kan arbejde i :D
hele mandag
tirsdag og onsdag Eftermiddag
Torsdag eller fredag Formiddag

\begin{itemize}
    \item \textbf{Cyber-Deception}
    \item Honeypots - Anders
    \item IPS (intrustion prevention system) Magnus
    \item IDS (intrusion detection system) Magnus
    \item ADS (Anomaly detection system) Magnus
    \item IOCs (indicators of compromise) - Betim
    \item DMZ (demilitarized zones) - Betim
    \item Game theory -Anders
    \item Quantum Computing Cryptography 
    \item Attackers og defenders - Betim 
    \item Active og passive deception - Anders
    \item cyber intruders, advanced persistent threats (APT), malicious insiders - Malthe
    \item signature-based detectors - Malthe
    \item masking,dazzling,mimicking and decoying - Malthe
    
\end{itemize}
\section{Introduction}
The area of cybersecurity deals with issues of protecting and encrypting sensitive and personal data and information available online through the use of the internet. Traditional methods of defending against the threat of malicious agents trying to gain access to this information have shown themselves inadequate in the face of more sophisticated attackers, with typically used defenses such as boundary controllers, firewalls and malware scanners having a decreasing effect on the new landscape of exploits and threats, presented by hackers. In a new world of social engineering and software exploits, malicious agents are able to install malware and backdoors with the intent of extracting vulnerable and sensitive data, such as credit card information and personnel information with impunity. \\\\
\noindent
The war that defensive cyber security experts wage with adversarial attacks is a losing one, as each time a new tecnological advance is made in the area of cyber security, attackers adapt and engage in new ways of circumventing these techonology-based defenses. Cyber deception and cyber denial present new and innovative ways to engage these attackers on the defenders own terms, and allows defenders to employ a more active cyber defense, in order to insure the safety and integrity of sensetive and important private data. The objective of this new approach to cyber securty, is to influence malicious attackers to behave in a way that gives the defenders of the system an advantage. This is done by trying to affect the attackers in such a way that a causal relationship can be established between the attackers psychological state and their behavior. Employing cyber deception means concealing what is true and false in the cyber domain, and creating a perpetual ambiguity of what the attacker perceives, which prevents them from accurately perceiving what is exploitable and vulnerable, and what is a trap laid by the security expert. Using this method, the defender (deceiver's) goal is to contruct a sort of cover story which generates a false certainty of what is real and fake, making the attacker confident in their malicious actions, or to seed doubt and uncertainty in order to make the attacker waste time and resources on failed attempts at gathering information about the system.   


\section{Operationalizing cyber deception}
This chapter will outline some of the overarching ways cyber deception is employed as a means to prevent adversaries from infiltrating and extracting information from web based systems. The techniques used in cyber deception can be broken down into 3 approaches, Host-based techniques, Network-based techniques and Hybrid techniques (vincent E.). \\
Host-based deception techniques typically tend to employ the use of honeypot-based defences with the goal of luring adversaries into a sandbox environment in which the defender is able to decern the methods and intentions of the the attacker. The intricasies of using honey pots will be discussed in a later chapter, but in short, a honeypot is form of virtual trap that appears to contain important and sensitive information, whose true purpose is for the defender to gain introspective of the adversary. A related technique is the deployment of so called patch-based vulnerability hiding, in which a "honey-patch" is deployed in order to make the attacker believe that a server is still vulnerable after a patch has been deployed in order to fix a discovered vulnerablity. Honey-patching is however not commonly used, as it is inherently very difficult to fix a vilnerability and still indicate to an adversary, that the vulnerability still exists. Anothe example of the deception being used in this approch is throught the use of ghost patches, in which a software patch is being announced that contains both real and false vulnerability fixes. This is useful, as it is common for adversaries to attempt to reverse engineer vulnerability patches, in order for their exploits to be effective. By deploying ghost patches, the adversaries potentially waste a large amount of time on on reverse engineering fake fixes that indicate a perceived vulnerability that was never there to begin with. 
\\\\
\noindent
Network-based techniques work by deceiving adversaries with falsified entities within specefic device networks (Bringer et al, find artikel).  The aim is to make it difficult for adversaries to hide their attacks, by using software-defined networking and network-based solutions. An example of this approach could be monitoring attacks on routing protocols such as the Routing Information Protocol (RIP) or Open Shortest Path First (OSPF). \\
Hybrid techniques are characterized by involving a mix of both host-based and network-based techniques. The use of hybrid techniques often incorporates many different deceptive components such as masking and dazzling, in order to more effectively deceive adversaries. \\\\
\noindent

\section{Signature-based detectors}

\section{White/blacklisting}

\section{Honeypots}
Current research... din mors fisse

\section{Litterature}
- Deception book
- Technologies to enable cyber deception


\end{document}
