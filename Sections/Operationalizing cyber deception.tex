\section{Operationalizing cyber deception}
This chapter will outline some of the overarching ways cyber deception is employed as a means to prevent adversaries from infiltrating and extracting information from web based systems. The techniques used in cyber deception can be broken down into 3 approaches, Host-based techniques, Network-based techniques and Hybrid techniques (vincent E.). \\
Host-based deception techniques typically tend to employ the use of honeypot-based defences with the goal of luring adversaries into a sandbox environment in which the defender is able to decern the methods and intentions of the the attacker. The intricasies of using honey pots will be discussed in a later chapter, but in short, a honeypot is form of virtual trap that appears to contain important and sensitive information, whose true purpose is for the defender to gain introspective of the adversary. A related technique is the deployment of so called patch-based vulnerability hiding, in which a "honey-patch" is deployed in order to make the attacker believe that a server is still vulnerable after a patch has been deployed in order to fix a discovered vulnerablity. Honey-patching is however not commonly used, as it is inherently very difficult to fix a vilnerability and still indicate to an adversary, that the vulnerability still exists. Anothe example of the deception being used in this approch is throught the use of ghost patches, in which a software patch is being announced that contains both real and false vulnerability fixes. This is useful, as it is common for adversaries to attempt to reverse engineer vulnerability patches, in order for their exploits to be effective. By deploying ghost patches, the adversaries potentially waste a large amount of time on on reverse engineering fake fixes that indicate a perceived vulnerability that was never there to begin with. 
\\\\
\noindent
Network-based techniques work by deceiving adversaries with falsified entities within specefic device networks (Bringer et al, find artikel).  The aim is to make it difficult for adversaries to hide their attacks, by using software-defined networking and network-based solutions. An example of this approach could be monitoring attacks on routing protocols such as the Routing Information Protocol (RIP) or Open Shortest Path First (OSPF). \\
Hybrid techniques are characterized by involving a mix of both host-based and network-based techniques. The use of hybrid techniques often incorporates many different deceptive components such as masking and dazzling, in order to more effectively deceive adversaries. \\\\
\noindent
