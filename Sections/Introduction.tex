\section{Introduction}
The area of cybersecurity deals with issues of protecting and encrypting sensitive and personal data and information available online through the use of the internet. Traditional methods of defending against the threat of malicious agents trying to gain access to this information have shown themselves inadequate in the face of more sophisticated attackers, with typically used defenses such as boundary controllers, firewalls and malware scanners having a decreasing effect on the new landscape of exploits and threats, presented by hackers. In a new world of social engineering and software exploits, malicious agents are able to install malware and backdoors with the intent of extracting vulnerable and sensitive data, such as credit card information and personnel information with impunity. \\\\
\noindent
The war that defensive cyber security experts wage with adversarial attacks is a losing one, as each time a new tecnological advance is made in the area of cyber security, attackers adapt and engage in new ways of circumventing these techonology-based defenses. Cyber deception and cyber denial present new and innovative ways to engage these attackers on the defenders own terms, and allows defenders to employ a more active cyber defense, in order to insure the safety and integrity of sensetive and important private data. The objective of this new approach to cyber securty, is to influence malicious attackers to behave in a way that gives the defenders of the system an advantage. This is done by trying to affect the attackers in such a way that a causal relationship can be established between the attackers psychological state and their behavior. Employing cyber deception means concealing what is true and false in the cyber domain, and creating a perpetual ambiguity of what the attacker perceives, which prevents them from accurately perceiving what is exploitable and vulnerable, and what is a trap laid by the security expert. Using this method, the defender (deceiver's) goal is to contruct a sort of cover story which generates a false certainty of what is real and fake, making the attacker confident in their malicious actions, or to seed doubt and uncertainty in order to make the attacker waste time and resources on failed attempts at gathering information about the system.   

